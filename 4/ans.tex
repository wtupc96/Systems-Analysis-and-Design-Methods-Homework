\newpage
\paragraph{答:}
顶层数据流图如下:
\begin{minipage}{\textwidth}
	\includepdf{ContextDFD.pdf}
\end{minipage}

\newpage
细化顶层数据流图得第一层数据流图:
\begin{minipage}{\textwidth}
	\includepdf{1DFD.pdf}
\end{minipage}

\newpage
细化系统的主要功能得到第二层数据流图:
\begin{minipage}{\textwidth}
	\includepdf{2DFD.pdf}
\end{minipage}

\newpage
通过这次实验,我大致对绘制数据流图的过程(自顶向下,逐步求精)有了一个初步的体会,大致步骤如下:
\begin{enumerate}
	\item 确定系统边界,辨别系统内部和外部实体;
	\item 确定外部实体和系统之间的数据流;
	\item 细化系统得到更为细致的系统内部过程,若能抽出数据存储也将其画出;
	\item 再细化过程得到更细致的过程,直至得到原子过程。
\end{enumerate}
\textbf{在逐步求精的过程中,特别要注意的一点就是要保持父图与子图的平衡。也就是说,父图中的某过程的输入输出流必须与他的子图的输入输出数据流在数量上和名字上相同。值得注意的是,如果父图中的一个输入(输出)数据流对应于子图中的几个输入(输出)数据流,而子图中组成这些数据流的数据项的全体正好是父图中的这一个数据流,那么他们仍然算是平衡的。} \\

那么对于这个题目而言,按照步骤走下来的过程如下:
\begin{enumerate}
	\item 外部实体有:捐助者、教育单位、基金会的理事会;
	\item 
	\begin{itemize}
		\item 
		捐助者到系统存在捐助请求,其实我还考虑到到底有没有另一条叫做基金的数据量,但是后来明确看到这是一个基金管理系统,也就是说只是对基金的\textbf{数额}进行管理而且捐款是存入银行的,所以不存在“基金”这样的数据流;
		
		\item 
		教育单位到系统存在用款申请,同样道理,也不存在“基金”这样的数据流从系统指向教育单位;
		
		\item
		基金会的理事会就只存在从系统指向理事会的“财政状况报表”。
	\end{itemize}
	\item 将系统分解为三个主要功能:收入处理、支出处理、产生报表并画出它们之间的数据流;
	\item 进一步细化得到第二层数据流图。
\end{enumerate}