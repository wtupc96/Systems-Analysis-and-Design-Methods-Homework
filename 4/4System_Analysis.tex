\documentclass[UTF8]{ctexart}
\usepackage{graphicx}
\usepackage{float}
\usepackage{subfiles}
\usepackage{pdfpages}
\usepackage[utf8]{inputenc}
\usepackage[english]{babel}
\usepackage{geometry}
\geometry{
	a4paper,
	total={170mm,257mm},
	left=20mm,
	top=20mm
}
\usepackage{listings}

%opening
\title{系统分析与设计方法 \\ 作业 4}
\author{软件42 \\ 欧阳鹏程 \\ 2141601030 \\ 版权声明:Creative Commons BY-NC-ND}

\begin{document}

\maketitle

\begin{enumerate}
	\item 
	\begin{itemize}
		\item 某教育基金会捐助基金管理系统的基本功能如下:
		\begin{itemize}
			\item 由捐助者向基金会提出捐助请求,经身份确认后被接受,对捐助人进行登记并授予捐助证书,捐款存入银行;
			\item 有教育单位提出用款申请,在进行相应合法性校验和核对相应的捐助存款后做出支出;
			\item 每月给基金会的理事会一份财政状况报表,列出本月的收入和支出情况和资金余额。
		\end{itemize}
		\item 要求:
		\begin{itemize}
			\item 确定上述系统的数据源点和终点,画出该系统的顶层数据流图;
			\item 分析系统的主要功能,细化系统的顶层数据流图,画出系统的第一层数据流图;
			\item 细化系统的各个主要功能,画出系统的第二层数据流图。
		\end{itemize}
		\item 提示:
		\begin{itemize}
			\item 系统中有三个实体:捐助者,教育单位,基金会的理事会;
			\item 系统的主要功能有:收入处理、支出处理、产生报表。其中收入处理可以细化为:接受请求(捐助请求)、确认身份和登记收入(存入款项);支出处理可以细化为:接受请求(用款请求)、合法性检查和登记支出(支出款项);
			\item 系统需要存储的信息:捐助者信息、教育单位信息、收支情况信息。
		\end{itemize}
		\item 按照上述要求画出数据流图。
		\item 作业最后包含实验总结和体会。
	\end{itemize}
	\subfile{ans.tex}	
\end{enumerate}
\end{document}
